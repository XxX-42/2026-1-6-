% 1_background.tex - 课题背景与意义

\section{课题背景与意义}

\subsection{研究背景}
随着移动互联网的高速发展,\textbf{短视频平台用户规模已突破10亿},以抖音、快手、B站为代表的内容平台每日产生海量多模态数据。传统的\textbf{单模态情感分析}方法在面对复杂语义时暴露出显著局限性:

\begin{itemize}
    \item \textbf{文本模态局限}:纯文本难以准确识别\textbf{反讽}(如"真是太棒了"的负面含义)与\textbf{双关}等修辞手法,缺乏语气、表情等辅助信息。
    \item \textbf{视觉模态局限}:单纯的面部表情识别在非受控环境下易受光照、遮挡影响,且无法区分"职业微笑"与真实情感。
    \item \textbf{声学模态局限}:孤立的语音信号难以判断说话者的真实意图,尤其在背景噪声干扰时鲁棒性较差。
\end{itemize}

一个典型案例是中文语境下的\textbf{"笑哭"表情(😂)}:用户常以此表达无奈、嘲讽甚至悲伤,与字面"笑"的含义形成矛盾。单一信息源无法解析这种\textbf{跨模态语义冲突},导致情感判断失误。

\subsection{研究意义}
本课题通过融合\textbf{文本、视觉、声学三通道}信息,旨在解决上述单模态瓶颈,具有以下核心价值:

\begin{enumerate}
    \item \textbf{互补性验证}:当某一模态信息模糊时(如文字反讽),其他模态(如语气愤怒、表情严肃)可提供印证,消除歧义。
    \item \textbf{鲁棒性提升}:多通道冗余设计使模型在部分信息缺失(如视频无字幕、音频静音)时仍能维持基本判断能力。
    \item \textbf{应用场景广泛}:研究成果可直接应用于\textbf{舆情监控}(识别网络负面情绪)、\textbf{智能客服}(判断用户满意度)及\textbf{人机交互}(情感陪伴机器人)等领域。
\end{enumerate}
