% 2_research_status.tex - 国内外研究现状

\section{国内外研究现状}

\subsection{技术发展历程}
多模态情感分析的研究经历了三个主要阶段:

\begin{enumerate}
    \item \textbf{早期拼接阶段 (2015前)}:采用\textbf{Early Fusion}(特征级拼接)或\textbf{Late Fusion}(决策级融合)的简单策略,将各模态特征直接串联后送入分类器。该方法实现简单,但忽略了模态间的交互关系。
    
    \item \textbf{中期注意力阶段 (2015-2019)}:引入\textbf{双流网络}与\textbf{注意力机制},允许模型动态学习模态权重。代表性工作如 TFN (Tensor Fusion Network) 和 MFN (Memory Fusion Network) 尝试捕捉模态间的高阶交互。
    
    \item \textbf{当前大模型阶段 (2020至今)}:基于\textbf{Transformer}架构的预训练模型成为主流。\textbf{CLIP} (OpenAI) 实现了视觉-语言的跨模态对齐;\textbf{ViT} (Vision Transformer) 将图像建模为序列;\textbf{BERT}及其变体(如中文 RoBERTa)在文本理解上取得突破。这些预训练模型为下游多模态任务提供了强大的特征提取能力。
\end{enumerate}

\subsection{当前主要挑战}
尽管技术快速发展,多模态情感分析仍面临以下核心难题:

\begin{itemize}
    \item \textbf{模态对齐困难 (Alignment)}:文本、视频、音频的时序粒度不同(词级 vs 帧级 vs 音素级),如何实现精准的跨模态时序对齐仍是开放问题。
    \item \textbf{异构特征融合}:不同模态的特征空间差异巨大(如 BERT 输出 768 维语义向量,ResNet 输出 2048 维视觉特征),简单拼接可能导致维度爆炸或信息稀释。
    \item \textbf{中文数据集稀缺}:现有主流数据集(如 CMU-MOSI, CMU-MOSEI)以英文为主,\textbf{高质量中文多模态情感数据集极度匮乏},这直接制约了中文场景下的模型训练与评估。
\end{itemize}

上述挑战凸显了选择合适中文数据集的重要性,这也是本项目选用 \textbf{CH-SIMS} 数据集的核心动因。
