% 3_dataset.tex - 数据集介绍 (CH-SIMS)

\section{数据集介绍}

\subsection{CH-SIMS 数据集概述}
本项目选用 \textbf{CH-SIMS} (Chinese Single- and Multi-modal Sentiment Analysis) 数据集,这是\textbf{首个包含细粒度标注的中文多模态情感数据集},由清华大学于 ACL 2020 发布。

\begin{itemize}
    \item \textbf{数据来源}:非受控环境下的\textbf{影视片段}(电影、电视剧、综艺),涵盖多样化的说话人、场景和情感表达。
    \item \textbf{样本规模}:共 \textbf{2,281} 个标注样本,划分为训练集 (1,368) / 验证集 (456) / 测试集 (457)。
    \item \textbf{标注特点}:提供\textbf{多模态独立标注},即文本、视觉、声学三个模态分别有独立的情感标签,便于分析各模态的贡献度。
\end{itemize}

\subsection{数据集难点}
\begin{itemize}
    \item \textbf{类别非平衡}:正负情感样本分布不均,需引入 \textbf{Class Weights}(类别加权)或过采样策略以防止模型偏向多数类。
    \item \textbf{噪声干扰}:部分样本存在背景音乐、多人对话、画面遮挡等干扰因素,对模型鲁棒性提出较高要求。
\end{itemize}

\subsection{数据预处理流程}
为确保多模态特征的有效提取,本项目设计以下预处理管线:

\begin{enumerate}
    \item \textbf{时序对齐}:采用 CH-SIMS 官方提供的 \textbf{Word-level Alignment} 标注,将视频帧与音频片段对齐至文本词级粒度。
    
    \item \textbf{人脸检测与对齐}:
    \begin{itemize}
        \item 使用 \textbf{MTCNN} (Multi-task Cascaded Convolutional Networks) 检测人脸边界框及\textbf{5点关键点}(双眼、鼻尖、嘴角)。
        \item 基于关键点进行\textbf{仿射变换},将人脸归一化至统一尺寸 ($224 \times 224$),消除姿态与尺度差异。
    \end{itemize}
    
    \item \textbf{数据清洗}:
    \begin{itemize}
        \item 剔除\textbf{黑帧}(全黑或无有效内容的视频帧)。
        \item 移除\textbf{静音片段}(音频能量低于阈值的样本)。
        \item 过滤人脸检测失败的样本。
    \end{itemize}
\end{enumerate}
