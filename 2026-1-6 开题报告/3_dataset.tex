% 3_dataset.tex - 数据集介绍

\section{数据集介绍}

\subsection{CH-SIMS 数据集概述}

本项目选用 \textbf{CH-SIMS} (Chinese Single- and Multi-modal Sentiment Analysis) 数据集作为实验基准。该数据集由清华大学于 ACL 2020 发布 \cite{chsims},是\textbf{首个包含细粒度独立标注的中文多模态情感数据集},填补了中文多模态情感分析领域的数据空白。

\subsubsection{数据来源}

CH-SIMS 的样本来源于\textbf{非受控环境}下的影视片段,包括电影、电视剧、综艺节目等。与实验室采集的受控数据不同,这些真实场景样本具有以下特点:
\begin{itemize}
    \item 说话人多样:涵盖不同年龄、性别、口音的演员。
    \item 场景复杂:包含多人对话、背景音乐、光照变化等干扰因素。
    \item 情感表达自然:非刻意表演,接近日常生活中的情感流露。
\end{itemize}

\subsubsection{数据规模与划分}

\begin{table}[htbp]
\centering
\caption{CH-SIMS 数据集划分}
\begin{tabular}{lccc}
\toprule
\textbf{集合} & \textbf{训练集} & \textbf{验证集} & \textbf{测试集} \\
\midrule
样本数 & 1,368 & 456 & 457 \\
占比 & 60\% & 20\% & 20\% \\
\bottomrule
\end{tabular}
\end{table}

总样本量为 \textbf{2,281} 个视频片段,每个片段时长 2--8 秒,配有对应的文字转录、视频帧与音频轨道。

\subsubsection{标注特点}

CH-SIMS 提供\textbf{多模态独立标注},即文本、视觉、声学三个模态分别有独立的情感标签(正向/中性/负向)。这一设计使研究者能够:
\begin{itemize}
    \item 分析各模态对最终情感判断的\textbf{独立贡献}。
    \item 研究模态间的\textbf{一致性与冲突}现象(如文字积极但语气消极)。
\end{itemize}

\subsection{数据集难点}

\begin{enumerate}
    \item \textbf{类别非平衡}:正、中、负三类样本分布不均,需引入 \textbf{Class Weights}(类别加权损失)防止模型偏向多数类。
    
    \item \textbf{噪声干扰}:部分样本存在背景音乐、多人重叠说话、人脸遮挡等问题,对模型鲁棒性提出较高要求。
    
    \item \textbf{模态缺失}:少量样本存在单模态信息缺失(如无字幕、静音片段),需在预处理阶段进行筛选或填充。
\end{enumerate}

\subsection{数据预处理流程}

为确保多模态特征的有效提取,本项目设计以下\textbf{三阶段预处理管线}:

\subsubsection{时序对齐}

采用 CH-SIMS 官方提供的 \textbf{Word-level Alignment} 标注文件,将视频帧序列与音频片段精确对齐至文本的词级粒度。对齐信息记录了每个词对应的起止时间戳,便于后续按词切分视觉与声学特征。

\subsubsection{视觉预处理}

\begin{enumerate}
    \item \textbf{人脸检测}:使用 \textbf{MTCNN} (Multi-task Cascaded Convolutional Networks) 检测视频帧中的人脸边界框。
    
    \item \textbf{关键点定位}:提取人脸的 \textbf{5 点关键点}(左眼中心、右眼中心、鼻尖、左嘴角、右嘴角),用于后续对齐。
    
    \item \textbf{仿射变换}:基于关键点计算仿射变换矩阵,将人脸归一化至 $224 \times 224$ 的标准尺寸,消除姿态、尺度差异。
    
    \item \textbf{黑帧剔除}:检测并移除全黑或无有效内容的帧(通过计算帧像素方差,低于阈值则判定为无效帧)。
\end{enumerate}

\subsubsection{声学预处理}

\begin{enumerate}
    \item \textbf{特征提取}:使用 \textbf{OpenSMILE} \cite{opensmile} 工具包提取标准化的底层声学特征,包括:
    \begin{itemize}
        \item \textbf{MFCC}(梅尔频率倒谱系数):13 维静态系数 + 一阶差分 + 二阶差分 = 39 维
        \item \textbf{Chroma}(色度特征):12 维,反映音高分布
        \item \textbf{Energy}(能量)、\textbf{Pitch}(基频)等韵律特征
    \end{itemize}
    
    \item \textbf{静音过滤}:检测音频能量低于阈值的静音片段,并进行标记或剔除。
    
    \item \textbf{特征归一化}:对提取的声学特征进行 Z-Score 标准化,消除量纲差异。
\end{enumerate}
