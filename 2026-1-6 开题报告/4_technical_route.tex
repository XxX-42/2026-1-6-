% 4_technical_route.tex - 核心技术路线

\section{核心技术路线}

\subsection{总体架构}
本研究采用\textbf{三塔架构 (Three-Tower Architecture)} 进行多模态特征提取,并通过\textbf{Early Fusion(特征级拼接)}实现模态融合。该方案的核心优势在于:结构简洁、易于调试、显存占用可控,适合在有限硬件条件下完成毕业设计。

整体流程为:\textbf{三路独立编码} $\rightarrow$ \textbf{特征拼接} $\rightarrow$ \textbf{全连接分类}。

\subsection{文本塔 (Text Tower)}
\begin{itemize}
    \item \textbf{模型选择}:采用 \texttt{bert-base-chinese} 预训练模型(HuggingFace 官方发布)。
    \item \textbf{特征提取}:将输入文本经 Tokenizer 编码后送入 BERT,取最后一层的 \textbf{[CLS] token} 作为句级语义表示。
    \item \textbf{输出维度}:$\mathbf{768}$ 维向量。
\end{itemize}

\subsection{视觉塔 (Visual Tower)}
\begin{itemize}
    \item \textbf{人脸预处理}:使用 \textbf{MTCNN} 进行人脸检测,提取\textbf{5点关键点}(双眼中心、鼻尖、左右嘴角),通过仿射变换完成人脸对齐,输出 $224 \times 224$ 归一化图像。
    \item \textbf{特征提取}:将对齐后的人脸帧序列送入预训练的 \textbf{ResNet-50}(在 ImageNet 上预训练),提取每帧的 $2048$ 维特征向量。
    \item \textbf{时序聚合}:对视频片段内的所有帧特征进行\textbf{时序均值池化 (Temporal Mean Pooling)}:
    \begin{equation}
        \mathbf{V} = \frac{1}{N} \sum_{i=1}^{N} \mathbf{v}_i
    \end{equation}
    其中 $N$ 为帧数,$\mathbf{v}_i$ 为第 $i$ 帧的 ResNet 输出。
    \item \textbf{输出维度}:$\mathbf{2048}$ 维向量。
\end{itemize}

\subsection{声学塔 (Acoustic Tower)}
\begin{itemize}
    \item \textbf{特征提取}:使用 \textbf{OpenSMILE} 工具包提取底层声学特征,包括:
    \begin{itemize}
        \item \textbf{MFCC}(梅尔频率倒谱系数):13维 + $\Delta$ + $\Delta\Delta$ = 39维
        \item \textbf{Chroma}(色度特征):12维
        \item \textbf{Energy}(能量)、\textbf{Pitch}(基频)等韵律特征
    \end{itemize}
    \item \textbf{时序编码}:将帧级声学特征送入 \textbf{Conv1D} 卷积层进行局部时序建模。
    \item \textbf{聚合策略}:经卷积后进行 \textbf{MeanPool},输出固定长度向量。
    \item \textbf{输出维度}:$\mathbf{256}$ 维向量。
\end{itemize}

\subsection{融合与分类}
\begin{itemize}
    \item \textbf{特征拼接}:将三塔输出直接拼接 (Concatenation):
    \begin{equation}
        \mathbf{F} = [\mathbf{T}; \mathbf{V}; \mathbf{A}] \in \mathbb{R}^{768+2048+256} = \mathbb{R}^{3072}
    \end{equation}
    \item \textbf{正则化}:引入 \textbf{Dropout($p=0.3$)} 防止过拟合。
    \item \textbf{分类层}:拼接特征经全连接层 (FC) 映射至类别数,输出层使用 \textbf{Softmax} 激活。
    \item \textbf{损失函数}:\textbf{CrossEntropyLoss},结合 Class Weights 处理类别不平衡。
\end{itemize}

\subsection{硬件可行性与显存优化}
本项目的硬件环境为 \textbf{RTX 3060 Laptop GPU (6GB 显存)},为确保模型可在该配置下顺利训练,采用以下优化策略:

\begin{enumerate}
    \item \textbf{梯度累积 (Gradient Accumulation)}:将大 Batch 拆分为多个小 Batch,累积梯度后统一更新参数,等效增大批量而不增加显存峰值。
    \item \textbf{冻结骨干网络 (Freeze Backbone)}:冻结 BERT 和 ResNet 的底层参数,仅微调顶层,大幅减少可训练参数量与显存占用。
    \item \textbf{混合精度训练 (FP16)}:使用 PyTorch AMP 自动混合精度,在保持精度的前提下降低显存消耗约 30\%--50\%。
\end{enumerate}
