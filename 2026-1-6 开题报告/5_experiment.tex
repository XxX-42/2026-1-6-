% 5_experiment.tex - 实验设计与预期产出

\section{实验设计与预期产出}

\subsection{消融实验设计}

为系统验证各模态的\textbf{独立贡献}及多模态融合的\textbf{互补效应},本项目设计以下 \textbf{7 组消融实验 (Ablation Study)}:

\begin{table}[htbp]
\centering
\caption{消融实验配置}
\begin{tabular}{clll}
\toprule
\textbf{序号} & \textbf{实验组} & \textbf{模态组合} & \textbf{验证目标} \\
\midrule
1 & Baseline-T & 仅文本 (Text) & 文本模态独立性能上界 \\
2 & Baseline-V & 仅视觉 (Visual) & 面部表情对情感判断的贡献 \\
3 & Baseline-A & 仅声学 (Acoustic) & 语音韵律对情感判断的贡献 \\
\midrule
4 & Fusion-TV & 文本 + 视觉 & T-V 互补性验证 \\
5 & Fusion-TA & 文本 + 声学 & T-A 互补性验证 \\
6 & Fusion-VA & 视觉 + 声学 & V-A 互补性验证(无文本基线) \\
\midrule
7 & Full Model & 文本 + 视觉 + 声学 & 三模态融合最优性能 \\
\bottomrule
\end{tabular}
\end{table}

\subsection{实验假设}

基于多模态互补性理论,本项目提出以下\textbf{预期假设}:

\begin{enumerate}
    \item \textbf{H1(文本主导假设)}:在单模态实验中,Baseline-T 将取得最高性能,因为文本语义信息最为直接。
    
    \item \textbf{H2(模态互补假设)}:任意双模态组合的性能将\textbf{优于}各自的单模态基线,证明模态间存在互补效应。
    
    \item \textbf{H3(三模态最优假设)}:Full Model 将达到实验中的\textbf{最优性能},验证三通道信息融合的综合价值。
\end{enumerate}

\subsection{评估指标}

\begin{itemize}
    \item \textbf{Accuracy (准确率)}:
    \begin{equation}
        \text{Accuracy} = \frac{\text{正确预测数}}{\text{总样本数}}
    \end{equation}
    衡量模型的整体分类正确率。
    
    \item \textbf{Weighted F1 (加权 F1 值)}:
    \begin{equation}
        \text{Weighted F1} = \sum_{c=1}^{C} \frac{n_c}{N} \cdot F1_c
    \end{equation}
    其中 $n_c$ 为类别 $c$ 的样本数,$N$ 为总样本数。该指标对各类别的 F1 按样本数加权平均,\textbf{更适合评估非平衡数据集}。
\end{itemize}

\subsection{实验环境}

\begin{table}[htbp]
\centering
\caption{实验环境配置}
\begin{tabular}{ll}
\toprule
\textbf{组件} & \textbf{规格} \\
\midrule
GPU & NVIDIA RTX 3060 Laptop (6GB) \\
CPU & Intel Core i7-12700H \\
内存 & 16GB DDR5 \\
操作系统 & Windows 11 / Ubuntu 22.04 (WSL2) \\
深度学习框架 & PyTorch 2.0+ \\
预训练模型 & bert-base-chinese, ResNet-50 (ImageNet) \\
\bottomrule
\end{tabular}
\end{table}

\subsection{预期产出}

\begin{enumerate}
    \item \textbf{工程代码}:
    \begin{itemize}
        \item 基于 PyTorch 的完整多模态情感分析系统源码。
        \item 包含数据预处理、模型定义、训练脚本、评估脚本。
        \item 代码结构清晰,配有详细注释与 README 文档。
    \end{itemize}
    
    \item \textbf{可视化演示}:
    \begin{itemize}
        \item 提供基于 Gradio 或 Streamlit 的简易 Demo 界面。
        \item 支持上传短视频片段,实时输出情感预测结果与置信度。
    \end{itemize}
    
    \item \textbf{学术论文}:
    \begin{itemize}
        \item 包含完整消融实验分析的本科毕业设计论文。
        \item 论文结构符合学校规范,涵盖背景、方法、实验、结论等章节。
    \end{itemize}
\end{enumerate}
