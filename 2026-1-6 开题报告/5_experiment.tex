% 5_experiment.tex - 实验设计与预期产出

\section{实验设计与预期产出}

\subsection{消融实验设计}
为验证各模态的独立贡献及多模态融合的有效性,本项目设计以下\textbf{消融实验 (Ablation Study)}:

\begin{table}[htbp]
\centering
\caption{消融实验配置}
\begin{tabular}{llc}
\toprule
\textbf{实验组} & \textbf{模态组合} & \textbf{验证目标} \\
\midrule
Baseline-T & 仅文本 (Text) & 文本模态独立性能 \\
Baseline-V & 仅视觉 (Visual) & 视觉模态独立性能 \\
Baseline-A & 仅声学 (Acoustic) & 声学模态独立性能 \\
\midrule
Fusion-TV & 文本 + 视觉 & T-V 互补性验证 \\
Fusion-TA & 文本 + 声学 & T-A 互补性验证 \\
Fusion-VA & 视觉 + 声学 & V-A 互补性验证 \\
\midrule
Full Model & 文本 + 视觉 + 声学 & 三模态融合最优性能 \\
\bottomrule
\end{tabular}
\end{table}

\subsection{评估指标}
\begin{itemize}
    \item \textbf{Accuracy (准确率)}:正确分类样本占总样本的比例,衡量整体分类性能。
    \item \textbf{Weighted F1 (加权 F1 值)}:考虑类别不平衡的综合指标,对各类别的 F1 按样本数加权平均,更适合非平衡数据集评估。
\end{itemize}

\subsection{预期实验结论}
\begin{enumerate}
    \item 单模态中,\textbf{文本模态 (Baseline-T)} 预期表现最优,因语义信息最为直接。
    \item 双模态融合相比单模态有显著提升,验证模态间的\textbf{互补性假设}。
    \item \textbf{Full Model(三模态融合)}预期达到最优性能,证明三通道信息的综合价值。
\end{enumerate}

\subsection{预期产出}
\begin{enumerate}
    \item \textbf{工程代码}:基于 PyTorch 的完整多模态情感分析系统源码,包含数据预处理、模型定义、训练与评估脚本。
    \item \textbf{可视化演示}:提供简易 Demo 界面,支持输入视频片段并输出情感预测结果。
    \item \textbf{学术论文}:包含消融实验分析、性能对比及结论讨论的本科毕业设计论文。
\end{enumerate}
