% 6_schedule.tex - 进度安排

\section{进度安排}

本毕业设计计划于 2026 年 1 月至 5 月期间完成,整体进度安排紧凑且目标明确。

第一阶段为\textbf{数据准备与预处理期}(1 月至 2 月)。作为项目的基石,本阶段的核心任务是确保数据处理流程的稳健性。这包括:实现基于 MTCNN 的人脸检测与关键点定位脚本,确保在 CH-SIMS 数据集上能够稳定运行;利用 OpenSMILE 批量提取声学特征并进行标准化处理;解析官方对其文件,完成三模态数据的时序对齐。该阶段的完成标志是建立一套可复现的、无错误的预处理管线。

第二阶段为\textbf{模型构建与基线验证期}(3 月)。工作重点将转向模型的具体实现。我们将首先构建单模态基线(特别是文本模态),验证数据流的可行性。随后,逐步接入视觉和声学模块,构建完整的三塔架构。在此期间,显存优化策略(如冻结骨干、梯度累积、混合精度训练)将是调试的重点,以确保模型能在硬件限制下正常训练。本阶段的目标是实现三模态模型的端到端训练,并观察到验证集损失的有效下降。

第三阶段为\textbf{实验与优化期}(4 月)。此阶段将集中进行系统的消融实验,验证各模态及融合策略的有效性。我们将记录并分析各实验组的 Accuracy 和 Weighted F1 指标,绘制性能对比图表。此外,如有余力,将尝试微调 BERT 更多层数等策略以探索性能上限。

第四阶段为\textbf{论文撰写与答辩准备期}(5 月)。根据实验数据和研究成果,撰写学位论文,内容涵盖研究背景、方法论、实验设计及结果分析等。同时,制作答辩演示文稿(PPT)及可视化的系统演示 Demo,确保能够直观展示研究成果。最后,对项目代码进行整理和文档化,以便于后续的存档与交流。
