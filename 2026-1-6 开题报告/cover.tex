% cover.tex - 封面
\begin{titlepage}
    \vspace*{1cm}
    \begin{center}
        % 校徽 - 宽度适中
        \includegraphics[width=10cm]{D:/Documents/Codes/2025_Cuit_chinese-multimodal-emotion/2025_Cuit_chinese-multimodal-emotion_ppt&paper/image copy.png} \\
        \vspace{1cm}
        
        % 一级标题
        {\songti \zihao{2} 本科毕业论文(设计)} \\
        \vspace{1.5cm}
        
        % 二级标题 - 加粗,大号黑体,字间距拉大
        {\heiti \zihao{1} \textbf{开\quad 题\quad 报\quad 告}} \\
        \vspace{3cm}
        
        % 信息栏
        \zihao{3} \songti
        \renewcommand{\arraystretch}{1.6} % 稍微拉大行距
        \begin{tabular}{lc}
             % 使用 \makebox[width][s] 实现分散对齐
             % 这里的宽度 7em 足以容纳 "毕业设计题目" (6字) 并让短词分散对齐
            \makebox[7em][s]{毕业设计题目} & \underline{\makebox[18em][c]{基于多模态情感分析的研究和应用}} \\
            \makebox[7em][s]{学生姓名} & \underline{\makebox[18em][c]{梁嘉轩}} \\
            \makebox[7em][s]{学号} & \underline{\makebox[18em][c]{2022053029}} \\
            \makebox[7em][s]{所在学院} & \underline{\makebox[18em][c]{计算机学院}} \\
            \makebox[7em][s]{专业} & \underline{\makebox[18em][c]{计算机科学与技术}} \\
            \makebox[7em][s]{班级} & \underline{\makebox[18em][c]{计算机科学与技术221}} \\
            \makebox[7em][s]{指导教师} & \underline{\makebox[18em][c]{冯翱}} \\
        \end{tabular}
    \end{center}
    
    \vfill
    \begin{center}
        {\songti \zihao{3} 2026 年 1 月 \\ 成都信息工程大学 计算机学院}
    \end{center}
\end{titlepage}
